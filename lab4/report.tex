\documentclass[12pt,preprint]{aastex}
\usepackage{amsmath, amsfonts, amssymb}
\usepackage[margin=0.85in]{geometry}
\begin{document}

\title{The Galactic Plane: The Great Circle at Latitude Zero}

\author{Conor E. Stanton \\ \today}
\begin{abstract} 
Write an amazing abstract.
\end{abstract}

\tableofcontents
\clearpage

\section{Introduction}
Write an amazing introduction

\section{Terminology, Equipment, Theoretical Background}
\subsection{Terminology}
\begin{itemize}
\item \textit{Galactic Coordinates}: Definition. 
\item \textit{Galactic Plane}: Definition. Include the RA
  and DEC boundaries of our object. With the following plot, we could
  figure out when our object of interest was up. See diagram below.

%\begin{figure}[h!]                                                            
%\begin{center}                                                                
%\includegraphics[height=?\textheight,width=?\textwidth]{file.ps}               %\caption{Text.}                                                 
%\end{center}                                                                  
%\end{figure}

\item \textit{HI Line}: Definition.
\item \textit{Solar Circle}: Definition. Mention the 8.5 kpc.
\item \textit{Spiral Galaxy}: Definition.

\end{itemize}

\subsection{Equipment}
Explain Leuschner. Include a Picture. Include Sample Commands

\subsection{Theoretical Background}
Explain any Mathematics. Explain colors if necessary.

Eqn. 8 from the lab packet,
\begin{equation}
V_{Dopp}=\left[{V(R) \over R} - {V(R_{\odot}) \over
  R_{\odot}}\right] R_{\odot}sin(l) 
\end{equation}

\section{The Galactic Rotation Curve}
\subsection{Determining the Rotation Curve}


\subsection{Results}
\underline{Estimating $M_{grav}$ of the Galaxy within the Solar Circle}

Explain how you can estimate it. Furthermore, include the equation 
$v^{2} = {GM_{grav} \over R}$

Also get the radial dependence $M_{grav}(R)$?

\underline{Estimating $M_{gas}(R)$ of the Galaxy}

This is the gaseous mass. Also need to answer ``What fraction of the
total Galactic mas comes from interstellar gas in this section.''

\section{The Galactic Center}
\underline{Overview}

There is strong evidence for a supermassive black hole at the center of
our galaxy. The location of the supposed singularity is referred to as
Sagittarius $A^{*}$, and many data have been taken on the region. For
this part of our lab, we need to use spatial extents and velocities to
estimate this black hole's mass. In order to get an accurate estimate,
however, it is necesary to look at motions of matter lying really close
to the black hole. Unfortunately, we do not have the angular resolution for this, but we can make an estimate!

\underline{Results}

\section{Spiral Structure of the Galaxy}

How can we prove that our galaxy is spiral? Not all galaxies are spiral,
but why is the Milky Way one? For this part of the lab, we sought to
answer these questions. To detect spiral structure, we needed to make a
position-velocity plot along the Galactic plane [the image
  $T_{A}(l,V_{LSR})$ for $b=0^{\circ}$]. 

Do we see spirals in our position-velocity plot?
%\begin{figure}[h!]                                                            
%\begin{center}                                                                
%\includegraphics[height=?\textheight,width=?\textwidth]{file.ps}              
%\caption{Text.}                                                 
%\end{center}                                                                  
%\end{figure}

We consequently had to make a model of a spiral arm and see how it
projects onto position-velocity space. Since we know our rotation curve,
we can do this! We measured this for the inner galaxy (the solar
circle), and for simplicity, we assume in this lab that the rotation
curve is constant beyond the  solar circle; that is, [$V(R)=220 km/s for
  R>8.5kpc$].

In polar coordinates, the equation for a spiral arm is

\begin{equation}
R_{arm} = R_{0}e^{\kappa(\Phi-\Phi_{0})}
\end{equation}

where $R_{0}$, $\kappa$, and $\Phi_{0}$ are free paramters; $\kappa$ is
the tangent of the pitch angle. By projecting this into
position-velocity space using equation 8 in the lab packet and our knowledge of the
Galactic geometry we obtain the following diagram,

%\begin{figure}[h!]                                                            
%\begin{center}                                                                
%\includegraphics[height=?\textheight,width=?\textwidth]{file.ps}              
%\caption{Text.}                                                 
%\end{center}                                                                  
%\end{figure}

As a side note, comment on why finding spiral arms inside the Solar
circle is very difficult, and has thwarted efforts by astronomers over
the past several dedcades. 

\section{Conclusion}

\end{document}
